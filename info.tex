# Python Design Pattern
This tutorial will discuss what Design Pattern is and how we can implement using the Python programming language. We will also explain several types of design patterns. We will explore the different approaches to solve the Python problems. Python
is a high-level, dynamic type language and popularly used in almost every possible technical domain.


# What is Design Pattern?
The design pattern is a technique which used by the developer to solve the commonly occurring software design. In simple word, it is a predefine pattern to solve a recurring problem in the code. These patterns are mainly designed based on requirements analysis.

The design pattern is a part of the software development. It is a general repeatable solution for the potential problem in software development. We can follow the patterns details and apply a solution which suits our code.

We may often confuse the patterns and algorithm, but both are separate approaches to solve repetitive problems. Algorithms generally define the clear set of the solution that can be implemented in some problems, where the patters are high-level description of the solution.


For example - An algorithm is like a cooking recipe: we have a clear set of ingredients (or set of solutions) to cook something (problems or goals). On the other side, a pattern is like a blueprint: we can see what the result and its features are, but we can modify the order of implementation.

# Configuration of Design Pattern
In the below diagram, we describe the basic structure of the design pattern documentation. It is focused on what technology we are using to solve the problems and in what ways.

# In the above diagram-

# Pattern Name - It is used to define the pattern in a shortly and effectively.

 # Intent/ Motive - It defines the goal or what the pattern does.

 # Applicability - It defines all the possible areas where the pattern is applicable.

 # Participants and Consequences -It consists of classes and objects used in the design pattern with the list of consequences that exist with the pattern.


 # History of Patterns
Design patterns are the set of the solution of common problems in object-oriented design. When the problem's solution is repeated again and again in the various projects, someone eventually puts a name and defines the solution in detail. That is how the pattern gets recognized.


## Christopher Alexander
 has described the concept of the pattern for the first time in the book named A Pattern Language: Towns, Building, and Construction.


 This book defines a 'language' for designing the urban environment. The language is nothing but the patterns.

The four authors: Erich Gamma, John, Vlissiders, Ralph Johnson, and Richard Helm, were picked the idea of a pattern language. Later, they published the book named Design Patterns: Elements of Reusable Object-Oriented Software. This book contains the concept of design patterns using the programming language.

The book featured the 23 useful various problems of object-oriented designs; It gained the much popularity among the programmers and became the best seller book very quickly.

Interesting Fact - This book has a very long name so people started to call it "The book of gang of four" which was soon summarized to simply "The GoF book".

Many other object-oriented patterns are discovered after this book. Very soon, the pattern approach became very famous in the programming fields. There are many others patterns available apart from object-oriented design as well.


# Advantages of Using Design Pattern
The advantages of using the design patterns are given below.

 # All design patterns are language neutral.
 # Patterns offer programmers to select a tried and tested solution for the specific problems.
# It consists of record of execution to decrease any technical risk to the projects.
# Patterns are easy to use and highly flexible.
# Design Pattern in Python
# We are all familiar with Python's feature; if someone does not, let's have a brief introduction - Python is a high-level, open-source, and dynamic typed language. It has English-like syntax and easy to learn. It provides numerous libraries that support a variety of designs.


We are listed below the design patterns that are supported by Python. We will use these design patterns in our tutorial.

Model View Controller Pattern
* Flyweight Pattern
*   Factory pattern
*   Singleton pattern
*   Object Oriented Pattern
*   Strategy Pattern
*   Command Pattern
*   Chain of Responsibility Pattern
*   Abstract Factory Pattern
*   Proxy Pattern
*   Facade Pattern
*   Observer Pattern
*   Prototype Pattern
*   Template Pattern
*   Adapter Pattern
*   Builder Pattern
*   Prototype Pattern
*   Decorator Pattern
*   State Pattern


# Importance of Learn Design Pattern

Many of the software developers might work for many years without knowing about any single pattern. It can also happen we might be implementing a pattern without even knowing it. So, here the question arises, why should we learn the design pattern? Let's look at the following points, which light up the importance of design patterns in development.

Design patterns have the predefined set of tried and tested solutions to a common problem encountered while developing software. If we know about the design pattern, then we can apply the solution without wasting time. It also teaches us how to solve the problem using the principle of object-oriented design.
Design pattern also enhances the common understanding between the developer and their teammates. Suppose there is a problem in the code, and you can say "Use Singleton for that," and everyone can understand if he/she knows the design pattern and its name.
Design patterns are also useful for the learning purpose because they introduce the common problem that we may have ignored. They also allow thinking that area that may not have had the hand-on experience.


# Singleton Design Pattern in Python

Singleton design pattern is one of the Credential Design Pattern and it is easiest to implement. As the name describes itself, it is a way to provide one object of a particular type. It is used to describe the formation of a single instance of class while offering a single global access point to the object.

It prevents the creation of multiple objects of the single class. The newly created object will be shared globally in an application.

We can understand it with the simple example of Data connectivity. While setting up the database connection, we generate an exclusive Database connection object to work with the Database. We can perform every operation regarding database using that single connection object. This process is called a Single design pattern.


# Motivation
Singleton design patterns are specially used in application types that need mechanisms over access to a mutual resource. As we have discussed earlier, a single class can be used to define a single instance.


One of the best benefits of using a singleton pattern is that we can restrict the shared resource and maintain integrity. It also prevents the overwriting in the current code by the other classes ensuing perilous or incompetent code. We can call the same object at multiple points of programs without worrying that it may be overwritten in the same points.


# Implementation
To implement the singleton pattern, we use the static method. We create the getInstance() method that can return the shared resources. When we call the static method, either it gives the unique singleton object or an error singling an instantiated object's existence.

It restricts to create the multiple objects of a defined class and maintain integrity.

We can take an example of the simple analogy - A county has a single central government that controls and accesses the country's operation. No one can create another government in a certain period.

We can implement this analogy using the singleton pattern.


In the above code, we have instantiated an object and stored it in a variable. We have also defined construction, which checks if there is another existing class; otherwise, it will raise an exception. We have then defined the static method named get_instance(), which returns the existing instance; if it is not available, then create it and return.

When we execute the script, the one GovInstance object is stored at a single point in the memory. When we create another object, it raises an exception.

Method - 2: Double Checked Locking Singleton Design Pattern

The synchronization of the threading is no longer in use because the object never is equal to the None. Let's understand the following example.


# Advantages of Singleton Patterns


This pattern provides the following advantages.

# A class created using the singleton pattern violates the Single Responsibility Principle, which means it can solve two problems simultaneously.
 # Single Pattern is difficult to implement in the multithreading environment because we need to ensure that the multithreading environment wouldn't create singleton objects several times.
# It makes the unit testing very hard because they follow the global state to an application.


# Disadvantages of Single Pattern
Single Patterns also contain few demerits which are given below.

# A class created using the singleton pattern violates the Single Responsibility Principle which means it can solve two problems at a single time.
# Single Pattern is difficult to implement in the multithreading environment, because we need to ensure that multithreading environment wouldn't create singleton object several times.
# It makes the unit testing very hard because they follow the global state to an application.

# Applicability of Design Pattern
We define the applicability of singleton design pattern as follows.

# In the project, where we need a firm control over the global variables, we must use the Singleton Method.
# Singleton patterns solves the most occurring problems such as logging, caching, thread pools, and configuration setting and often used in conjunction with the Factory design pattern.


# Factory Design Pattern
A factory design pattern is one of the types of Creational Design Pattern that allows us to use an interface or a class to create an object to instantiate. The factory method offers us the best way to create an object. In this method, objects are created without revealing the logic to the client. To create a new type of object, the client uses the same common interface.


# Problem
Suppose we have planned to create a website which provides the books in the different part of the country. The website's initial version only takes the order of the books, but as time passes, our website gains popularity now, we won't add more items to sales such as clothes and footwear. It is a very good idea, but what about the software developers. Now they have to change the whole code because most parts of the code engaged with the book's class, and developers have to change the entire codebase. It may lead to a messy code.


# Solution
In the solution, we use the special factory method to call the construction object instead of using straight forward object construction. Both method of creating objects are quite similar but they are called within the factory method.

For Example - Our selling items such as Books, Mobile, Cloths, and Accessories classes should include the purchase interface which will declare a method buy. Each class will implement these methods uniquely.



# Merits of using Factory Method
Below are the advantages of the factory method.

#Factory methods are very useful in adding new types of product without distributing the existing client code.
It avoids the tight coupling between the products and the creator classes and objects.
# Demerits of using Factory Method
The disadvantages of using factory methods are given below.
It will create the large number of small files or cluttering of the files.
The client might have the sub class to create a particular actual product object.


# Applicability
Its concept is same as the polymorphism, where we don't need to make changes in the client code. For example - Suppose we want to draw different shapes such as Rectangle, Square, Circle, etc. We can use the factory method to create the instance depending upon the user's input.

In a cab application, we can book a 1- wheeler, 2- wheeler, 3-wheeler, and 4-wheeler. Here the user can book any of the rides which he wants to book. With the help of factory method, we can create a class named Booking which will help us to create the instance that takes the user's input. So here the developer doesn't need to change the entire code to add the new facility.

Factory method removes the complex logical code that is hard to preserve. It also prevents us to change in codebase because modifying existing code can introduce the subtle bugs and change in the behavior.


# Where to use factory Methods
First we need to identify the situations where the factory method can be implemented. It can be used where an application depends upon the interface (product) to accomplish the task and there are multiple concrete implementations of that defined interface.

There are many problems that can be resolved using the factory method. We define few example which fits in this description.


# Replacing complex logical code

Generally, the code consists of logic like if/else/elif that is hard to maintain due to adding new paths with some requirements change.

Using the factory method, we can put each logical path's body into the various defined functions or classes with a common interface. The developer can provide the concrete implantation for the modification.


# Combining similar features under a common interface

Suppose we need to apply the specific filter to an image. The factory method will find the specific filter according to the user input. The factory method can apply the actual implementation.


# Supporting multiple implementations of the same features

A group of scientists need transformations of the satellite images one coordinate system to another. But a system has multiple algorithms to perform the different level of transformation. The application can allow the user to select an ideal algorithm. Factory method can implement firmly algorithm based on this option.


# Integrating related external series

A video streaming application wants to integrate the multiple external services. The application allows to the users to know where their video comes from. The Factory method creates the right integration based on a user preference.


# Conclusion
Factory method is a popular and widely used creational design pattern. It can be used in the various concrete implementation of an interface exist. Factory method removes the complex logic code that is hard to maintain. It also prevents us to modifying the existing code to support the new requirement.

